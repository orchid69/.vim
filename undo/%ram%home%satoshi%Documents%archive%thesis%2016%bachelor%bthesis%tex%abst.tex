Vim�UnDo�-��c�H��<U	`��\0��w��&#�9"Y1f_�"����"vY1�"\etitle{@Binarizing Convolutional Neural Networks in the Frequency Domain}\eauthor{Satoshi Miyake}\eabstract{BThere is an increasing demand for image recognition technology on Wembedded devices, such as face recognition in digital camera and pedestrian recognitionin automatic driving.hTraditional image recognition methods are based on a classifier such as a support vector machine (SVM), 7which requires handcrafted features for specific tasks.hOn the other hand, recent improvement on GPUs enables deep learning, a more general and accurate method.jAmong the deep learning methods, convolutional neural networks~(CNN) have high image recognition accuracy.^However, the computational cost of CNN is high since CNN requires many convolution operations.UIn CNN, convolution operation accounts for a major portion of the computational cost.dTherefore, simplifying the convolution operation leads to great reduction of the computational cost.XIn this thesis, binarized convolutional neural networks in the frequency domain~(FDBNN),Fwhich utilizes binarized weights in the frequency domain, is proposed.aFDBNN intends to simplify the operation of convolutions and reduce memory consumption of weights.ZFDBNN has three advantages. First, by computing convolutions as pointwise products in the Qfrequency domain, the number of multiplication on CNN can be reduced. Second, by abinarizing the weights used in the filters, the size of memory can be reduced. Finally, binarized[weights in the freqency domain can reproduce the original filters more similarly than that Sin the spatial domain. This characteristic makes the proposed method more accurate.QSimulation results of classification experiments on a hand-written digit dataset [indicate that FDBNN can reduce the number of multiplications and achieve the same accuracy and memory consumption as BNN.}5�_�"����"vY1e�Kembedded devices, such as face recognition in digital camera and pedestrianLrecognition in automatic driving.  Traditional image recognition methods areLbased on a classifier such as a support vector machine (SVM), which requires(handcrafted features for specific tasks.KOn the other hand, recent improvement on GPUs enables deep learning, a moreLgeneral and accurate method.  Among the deep learning methods, convolutionalIneural networks~(CNN) have high image recognition accuracy.  However, theEcomputational cost of CNN is high since CNN requires many convolutionNoperations.  In CNN, convolution operation accounts for a major portion of theNcomputational cost.  Therefore, simplifying the convolution operation leads to*great reduction of the computational cost.HIn this thesis, binarized convolutional neural networks in the frequencyLdomain~(FDBNN), which utilizes binarized weights in the frequency domain, isMproposed.  FDBNN intends to simplify the operation of convolutions and reduceOmemory consumption of weights.  FDBNN has three advantages. First, by computingIconvolutions as pointwise products in the frequency domain, the number ofOmultiplication on CNN can be reduced. Second, by binarizing the weights used inMthe filters, the size of memory can be reduced. Finally, binarized weights inOthe freqency domain can reproduce the original filters more similarly than thatIin the spatial domain. This characteristic makes the proposed method more	accurate.PSimulation results of classification experiments on a hand-written digit datasetLindicate that FDBNN can reduce the number of multiplications and achieve the/same accuracy and memory consumption as BNN.  }5��