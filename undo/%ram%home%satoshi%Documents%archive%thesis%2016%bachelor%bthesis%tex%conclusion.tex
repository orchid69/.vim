Vim�UnDo����i�,�]_���	(���gjkF��� Y1`_�"����"vY1_� l本論文では,CNNの演算コストおよび,メモリサイズ圧縮に向けて,周波数領域で6重みを二値化するCNNの手法を提案した.oCNNは画像認識分野において,高い認識精度を達成しているが,演算コストおよび,cフィルタの重み保存用のメモリサイズが膨大になるという問題があった.uこの問題により,消費電力およびメモリサイズの観点から,大規模なCNNをハードウェア<実装することが,非常に困難になっている.]本論文では,CNNの演算コストのうち,大部分を占める畳み込み演算を`簡略化することで,演算コストの削減とメモリサイズの削減を図った.d既存手法であるBNNや,FDCNNを参考に,周波数領域で重みを二値化するFDCNNをh提案した.FDCNNは,畳み込み演算を周波数領域に変換することで演算コストをc削減し,重みの二値化によりメモリサイズを圧縮することが可能である.bまたFDCNNは,従来の重み二値化CNNであるBNNと比較して,フィルタの再現度Nが高いことから,BNNよりも識別精度の向上が期待される.xまた,提案手法の中で,重みの二値化方法,フィルタのサイズ,フィルタの共役対称性,K勾配の計算方法などに着目し,複数の手法を提案した.iこれら複数の提案手法に対し,データセットの識別精度をそれぞれ比較した.rまた,従来のCNNやBNNとの識別精度比較も行い,周波数領域における重みの二値化が,<空間領域と同様に有用であることを示した.iまた,提案手法の一つが,従来のCNNから約0.86\%の認識精度低下に維持しつつ,TメモリサイズをBNNと同じサイズまで削減できることを示した.u今後の展望として,CNN全体を周波数領域で行うようなネットワーク構造を考えている.o現在の実装では,層ごとにFFTと逆FFTを繰り返しているが,層の出力で逆FFTをせず,r周波数領域のまま学習できるようなCNNが実現できれば,現在と比べて大幅に計算量をi削減できる.プーリング層については,周波数領域に置き換えるものがすでにbSpectral Pooling\cite{rippel2015spectral}として,提案されているので,活性化関数iの部分を周波数領域でうまく表現できれば,全てを周波数領域で演算できるCNN'が実現できると期待される.5�_�"����Y1�#d\chapter{結論} \label{chap:conclusion}本論文では,CNNの演算コストおよび,メモリfサイズ圧縮に向けて,周波数領域で重みを二値化するCNNの手法を提案した.uCNNは画像認識分野において,高い認識精度を達成しているが,演算コストおよび,フィxルタの重み保存用のメモリサイズが膨大になるという問題があった.この問題により,消u費電力およびメモリサイズの観点から,大規模なCNNをハードウェア実装することが,非!常に困難になっている.u本論文では,CNNの演算コストのうち,大部分を占める畳み込み演算を簡略化することでt,演算コストの削減とメモリサイズの削減を図った.既存手法であるBNNや,FDCNNを参考sに,周波数領域で重みを二値化するFDCNNを提案した.FDCNNは,畳み込み演算を周波数領x域に変換することで演算コストを削減し,重みの二値化によりメモリサイズを圧縮するこqとが可能である.またFDCNNは,従来の重み二値化CNNであるBNNと比較して,フィルタのW再現度が高いことから,BNNよりも識別精度の向上が期待される.xまた,提案手法の中で,重みの二値化方法,フィルタのサイズ,フィルタの共役対称性,x勾配の計算方法などに着目し,複数の手法を提案した.これら複数の提案手法に対し,デuータセットの識別精度をそれぞれ比較した.また,従来のCNNやBNNとの識別精度比較も行xい,周波数領域における重みの二値化が,空間領域と同様に有用であることを示した.まrた,提案手法の一つが,従来のCNNから約0.86\%の認識精度低下に維持しつつ,メモリサHイズをBNNと同じサイズまで削減できることを示した.u今後の展望として,CNN全体を周波数領域で行うようなネットワーク構造を考えている.r現在の実装では,層ごとにFFTと逆FFTを繰り返しているが,層の出力で逆FFTをせず,周u波数領域のまま学習できるようなCNNが実現できれば,現在と比べて大幅に計算量を削減kできる.プーリング層については,周波数領域に置き換えるものがすでにSpectralhPooling\cite{rippel2015spectral}として,提案されているので,活性化関数の部分を周u波数領域でうまく表現できれば,全てを周波数領域で演算できるCNNが実現できると期待される.5��