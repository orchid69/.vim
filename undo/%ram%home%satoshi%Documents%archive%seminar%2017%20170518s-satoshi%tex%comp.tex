Vim�UnDo�e�b�7�rX����� `
|�x�T>����Yw%_�����nvYw�n\begin{figure*}[tb]
\centering0\includegraphics[width=\linewidth]{fig/xnor.png}\caption{XNOR-Netの概要図}\label{fig:xnor}
\end{figure*}\subsection{評価} R本節では,CNN, BWNおよびXNOR-Netについてその精度比較を行う.)\paragraph{AP2による影響} \mbox{} \\V先に示したApproximate power-of-2 (AP2)について,通常のBatch NormalizationYとAP2によるBatch Normalizationの精度を比較した.結果を図\ref{fig:ap2}に	示す.\begin{figure}[t]
\centering2\includegraphics[width=\linewidth]{fig/ap2_ac.eps}(\caption{AP2による影響(精度)}\label{fig:ap2}\end{figure}\begin{figure}[t]
\centering2\includegraphics[width=\linewidth]{fig/ap2_lo.eps}(\caption{AP2による影響(損失)}\label{fig:ap2_lo}\end{figure}xグラフより,十分学習が進めば,AP2による影響は小さく,十分無視できることがわかる.#\paragraph{精度評価} \mbox{} \\]CNN, BWN, XNOR-Netについて表\ref{tb:net}のような構成で,ニューラルネットbのフレームワークTorch7により精度比較を行った.データセットはCifar-10を用いた.\begin{table}[t]  \centering  \caption{Network構造}  \begin{tabular}{lcccc} \hlineP    層種類・名称 & チャネル & フィルタ & ストライド \\ \hline,      data         & 3  & 28x28 & -       \\*      conv1        & 96 & 3x3   & 1    \\       bn           \\      ReLU         \\)      pooling      & -  & 2x2   & 2    \\+      conv2        & 256 & 3x3   & 1    \\       bn           \\      ReLU         \\+      conv3        & 384 & 3x3   & 1    \\       bn           \\      ReLU         \\+      conv4        & 384 & 3x3   & 1    \\       bn           \\      ReLU         \\+      conv5        & 256 & 3x3   & 1    \\       bn           \\      ReLU         \\(      pooling      & -  & 2x2   & 2   \\(      fc1          & 1024 & - & -    \\       bn           \\      ReLU         \\(      fc2          & 1024 & - & -    \\       bn           \\      ReLU         \\&      fc3          & 10 & - & -    \\       bn           \\      SoftMax      \\ \hline  \end{tabular}  \label{tb:net}\end{table}^評価結果を図\ref{fig:ac} \ref{fig:lo}に示す.通常のCNNが最も精度がよく,\epoch=20時点では,86.9\%,次に精度がよいのが,BWNで84.8\%,最も精度がb悪かったのが,XNOR-Netで78.1\%であった.入力,重みの二値化の関係から,妥当な順位である.\begin{figure}[htbp]
\centering.\includegraphics[width=\linewidth]{fig/ac.eps}%\caption{各実装に対する精度}\label{fig:ac}\end{figure}\begin{figure}[htbp]
\centering.\includegraphics[width=\linewidth]{fig/lo.eps}%\caption{各実装に対する損失}\label{fig:lo}\end{figure}R精度,およびメモリサイズなどをまとめたものを表\ref{tb:comp}に示す.\begin{table}[tbp]  \centering'  \caption{CNN, BWN, XNOR-Netの比較}  \begin{tabular}{lccc} \hline�      ネットワーク & \shortstack{メモリサイズ\\削減量} & \shortstack{畳み込み演算で\\使用する演算} & 精度(\%) \\ \hline2      CNN          & 1x  & +, -, x & 86.9       \\.      BWN          & ~32x & +, - & 84.8    \\ <      XNOR         & ~32x & XNOR,bitcount & 78.1   \\ \hline  \end{tabular}  \label{tb:comp}\end{table}�メモリサイズは,元の重みが32bitであった場合,重みの二値化によりサイズが1/32になる.また,畳み込み層で使用する演算�に関しては,入力値,重みともに二値化することで,XNORとbitcountのみで実現することができる.5�_�����vYw�~関しては,入力値,重みともに二値化することで,XNORとbitcountのみで実現することができる.5�_�����vYw ��5�_�����VYw$�\section{評価}5��